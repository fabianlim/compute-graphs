\title{Optimizing compute graphs}

\documentclass[12pt]{article}

\usepackage{pgf,tikz}

\author{author}

\begin{document}
\maketitle

\input{definitions.tex}


\section{Enumeration}

Assumption: Only care about functions $f$ that are multivariate polynomials,
e.g., 
\[
    f = f(X_1, X_2, \cdots, X_n).
\]
I am going to ignore division and minus for now. 
I also don't really know how to handle the coefficients, I am somehow magically
assuming they don't exist (you will see later).

Definition: Compute graph $\CompG$ is a ``representation'' of some function $f$.
Each function $f$ is ``represented'' by multiple compute graphs.
Consider the following ``recursive'' definition of $\CompG$:
A compute graph $\CompG$ is said to be $k$-computation of a multivariate
polynomial $f$, if $f$ satisifes
\[
    f = g(p_1, p_2, \cdots, p_k)
\]
where 
i) g is a $k$-variate polynomial, and 
ii) $p_i$ are polynomials in $X_1, \cdots, X_n$ that have $k$-computation
representations by ``children'' graphs of $\CompG$. 
The motivation for $k$-computations from practical standpoints, is that the
maximum number of mults/adder is $k$. 
You can also limit the number of adds but that is a simple generalization.
Since I ignore division the polynomials $p_i$ must have degree at most that of $f$.


The below computation graph is a $2$-computation representation of $f$.
\begin{tikzpicture}
\node {$f$} [grow'=up]
    child { node {$+$}
            child { node {$p_1$} }
            child { node {$p_2$} }
            child { node {$*$}
                child {node {$p_3$} }
                child {node {$p_1$} }
            }
        };
\end{tikzpicture}

I am interested enumerating all possible $k$-computations of some multivariate
polynomial $f$.
If we can enumerate, we can pick the one with some smallest cost.

General idea: Maintain a pool of multivariate polynomials. 
Build larger degree polynomials by ``combining'' them via $k$-variate polynomials.  
We have to limit the pool by quickly eliminating polynomials that we know cannot
build $f$. 
We can use the following observation to eliminate candidates.

Observation: If a polynomial $p$ can be used to build $f$, then any monomial of
$p$ must divide some monomial of $f$. 

Example: A trite one, $f = X_1^n$. In this case, any polynomial $p$ that is a
function of any $X_2, \cdots$ is eliminated.

Example: Consider $f = X_1 X_2 + X_2 X_3 + X_1 X_3$. In this case, any
polynomial $p$ that has multiple variates, e.g. $X_2^2 X_3$, is eliminated.

Example: Why this fails with minus. Very simple. We could have the following
where $q$ could be an arbitrary large degree polynomial.

\begin{tikzpicture}
\node {$f$} [grow'=up]
    child { node {$+$}
        child { node {$p_1 - q$} }
        child { node {$p_2 + q$} }
        };
\end{tikzpicture}
Similar bad things will happen with divide.

Example: Hardamard transform. This is the Kronecker power of the matrix
\[
\left[
    \begin{array}{cc}
        1 & 1 \\ 1 & -1
    \end{array}
\right]
\]
The $2 \times 2$ transform is given by the polynomials $X_1 + X_2$ and
$X_1 - X_2$. The $4 \times 4$ transform is 
\begin{eqnarray}
    X_1 + X_2 + X_3 + X_4 \\
    X_1 - X_2 + X_3 - X_4 \\
    X_1 + X_2 - (X_3 + X_4) \\
    X_1 - X_2 - (X_3 - X_4) 
\end{eqnarray}
The following $1$-computation graph represents (all rows of the) Kronecker transform 
(or actually any Kronecker matrix product). 
\begin{tikzpicture}
\node {$f_n$} [grow'=up]
    child { node {$+$}
        child { node {$f_{n-1}$} }
        child { node {$f_{n-1}$} }
        };
\end{tikzpicture}
where $f_n$ is a row of the $2^n \times 2^n$ Hardamard transform. 
We simply need to adjust for the coefficients $1/-1$. 
This particular computation graph will achieve the smallest depth
(i.e., $\log(n)$ for each row).
There are of course other compute graphs, but the polynomial that represents all
rows of the $2^n \times 2^n$ transform has the general form
\[
    f_n = \sum_{i=1}^{2^n} \prod_{j=1}^m a_{ij} X_i, 
\]
where the coefficients $a_{ij}$ come from the Kronecker product. 
To achieve $n \log(n)$ total operations, there must be term reuse (recall the
butterfly structures) so we would have to optimize $2^n$ compute graphs
simulataneously. 
In this case we only need to maintain the pool of univariate polynomials in
$X_1, X_2, \cdots, X_n$. 



\end{document}
