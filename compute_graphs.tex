\title{Optimizing compute graphs}

\documentclass[12pt]{article}

\usepackage{pgf,tikz}

\begin{document}
\maketitle

\newcommand{\CompG}{\mathcal{G}}


\section{Enumeration}

Assumption: Only care about functions $f$ that are multivariate polynomials,
e.g., 
\[
    f = f(X_1, X_2, \cdots, X_n).
\]
I am going to ignore division for now. 
I also don't really know how to handle the coefficients, I am somehow magically
assuming they don't exist (you will see later).

Definition: Compute graph $\CompG$ is a ``representation'' of some function $f$.
Each function $f$ is ``represented'' by multiple compute graphs.
Consider the following ``recursive'' definition of $\CompG$:
A compute graph $\CompG$ is said to be $k$-computation of a multivariate
polynomial $f$, if $f$ satisifes
\[
    f = g(p_1, p_2, \cdots, p_k)
\]
where 
i) g is a $k$-variate polynomial, and 
ii) $p_i$ are polynomials in $X_1, \cdots, X_n$ that have $k$-computation
representations by ``children'' graphs of $\CompG$. 
The motivation for $k$-computations from practical standpoints, is that the
maximum number of mults/adder is $k$. 
You can also limit the number of adds but that is a simple generalization.
Since I ignore division the polynomials $p_i$ must have smaller degree than $f$.

The below computation graph is a $2$-computation representation of $f$.
\begin{tikzpicture}
    \node {$f$} [grow=up]
        child { node {$+$}
                child { node {$p_1$}}
                child { node {$p_2$}}
                child { node {$*$}
                    child {node {$p_3$}}
                    child {node {$p_1$}}
                }
            };
\end{tikzpicture}

I am interested enumerating all possible $k$-computations of some multivariate
polynomial $f$.
If we can enumerate, we can pick the one with some smallest cost.

General idea: Maintain a pool of multivariate polynomials. 
Build larger degree polynomials by ``combining'' them via $k$-variate polynomials.  
We have to limit the pool by quickly eliminating polynomials that we know cannot
build $f$. 
We can use the following observation to eliminate candidates.

Observation: If a polynomial $p$ can be used to build $f$, then any monomial of
$p$ must divide some monomial of $f$. 

Example: A trite one, $f = X_1^n$. In this case, any polynomial $p$ that is a
function of any $X_2, \cdots$ is eliminated.

Example: Consider $f = X_1 X_2 + X_2 X_3 + X_1 X_3$. In this case, any
polynomial $p$ that has multiple variates, e.g. $X_2^2 X_3$, is eliminated.

\end{document}
